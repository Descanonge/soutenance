% chktex-file 8

\documentclass[11pt, french]{beamer}

\usepackage{soutenance}

\title{%
  Impact des Fronts sur le Phytoplancton\\
  dans la Région du Gulf Stream\\
  Quantifié par Imagerie Satellitaire
}

\author{Clément Haëck}

\direction{Marina Lévy et Laurent Bopp}

\institute{%
  Laboratoire d'Océanographie et du Climat\\Expérimentations et Analyses Numériques
}

\begin{document}

{
  \usebackgroundtemplate{
    \parbox[t][\paperheight]{\paperwidth}{
    \vfill\par
    \hfill
    \includegraphics[width=0.7\paperwidth, height=0.7\paperheight, keepaspectratio]{title_background.jpg}}
  }
  \begin{frame}[plain, noframenumbering]
    \titlepage%
  \end{frame}
}

%% Préambule
\begin{frame}{Phytoplancton ?}
\end{frame}

\begin{frame}{Problématiques}
\end{frame}

\section{Introduction}

\begin{frame}{Frontogénèse et remontée des nutriments}
  \centering
  \multigraph[width=\textwidth, height=0.8\textheight, keepaspectratio]{5}{front_circulation}
\end{frame}

\section{Région subtropicale -- Augmentation de la Chlorophylle dans les fronts}
\sectionframe{1}

\section{Région du Gulf-Stream -- Intensité des fronts}
\sectionframe{2}

\section{Région subpolaire -- Phénologie du bloom}
\sectionframe{3}

\begin{frame}{Chronométrer le bloom: démarrage et durée}
  \multigraph[width=\textwidth, height=0.8\textheight, keepaspectratio]{9}{phenologie_méthode}
\end{frame}

\begin{frame}{Différences de phénologie entre background et fronts}
  \multigraph[width=\textwidth, height=0.8\textheight, keepaspectratio]{4}{phenologie_lag}
\end{frame}

\end{document}
