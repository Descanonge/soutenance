% chktex-file 8

\documentclass[11pt, french]{beamer}

\usepackage{soutenance}

\AtBeginDocument{%
  \bookmark[named=FirstPage, level=subsection]{Title frame}%
}

\title{%
  Impact des Fronts sur le Phytoplancton\\
  dans la Région du Gulf Stream\\
  Quantifié par Imagerie Satellitaire
}

\author{Clément Haëck}

\direction{Marina Lévy et Laurent Bopp}

\institute{%
  Laboratoire d'Océanographie et du Climat\\Expérimentations et Analyses Numériques
}

\begin{document}

{
  \usebackgroundtemplate{
    \parbox[t][\paperheight]{\paperwidth}{
    \vfill\par
    \hfill
    \includegraphics[width=0.7\paperwidth, height=0.7\paperheight, keepaspectratio]{title_background.jpg}}
  }
  \begin{frame}[plain, noframenumbering]
    \titlepage%
  \end{frame}
}

%% Préambule
\begin{frame}{Qu'est-ce que le phytoplancton}
  photos du phyto pour montrer la biodiversité

  souligner son importance avec schéma (simplifié) de la chaine trophique et du puit de carbone
\end{frame}

\begin{frame}
  \frametitle{Observer le phytoplancton}
  mesures in-situ + modèles, réprendre le schéma de Lévy 2023 (la partie du haut)

  ouverture sur l'utilité des images satellites
\end{frame}

\begin{frame}
  \frametitle{Le phytoplancton depuis l'espace}
  schéma explicatif du principe des images sat de chloro
\end{frame}

\begin{frame}
  \frametitle{Répartition du phytoplancton}
  image sat (false colors) du phyto (fig 1 de la thèse)

  commentaires sur les différences entre
  \begin{itemize}
    \item sud et nord du GS (grande échelle)
    \item plus petites échelles (tourbillons, filaments, \alert{fronts})
  \end{itemize}
\end{frame}

\begin{frame}
  \frametitle{Influence des courants}
  courants grande échelle: répartition des nutriments + biomes (figure répartition nitrate)

  courants petite échelles
\end{frame}

\begin{frame}
  \boxtitle{Problématiques}
  \vspace{2em}

  \begin{enumerate}
    \setlength{\itemsep}{1em}
    \item Quantifier la \emph{réponse de la chlorophylle-a} aux dynamiques \emph{frontales} dans la région du Gulf Stream
    \uncover<2->{\item Influence de la \emph{saison}, du \emph{régime} biogéochimique, et de \emph{l’intensité} des fronts ?}
    \uncover<3->{\item Détecter un \emph{bloom précoce} dans les fronts ?}
  \end{enumerate}
\end{frame}

\section{Introduction}

\subsection{Impact des fronts sur le phytoplanctons}
\subsubsection{Upwelling de nutriments}

\begin{frame}
  \frametitle{Frontogénèse et remontée des nutriments}
  \centering
  \multigraph[width=\textwidth, height=0.8\textheight, keepaspectratio]{5}{front_circulation}
\end{frame}

\subsubsection{Modification de la phénologie du bloom}

\begin{frame}
  \frametitle{Déclenchement du bloom: hypothèse de Sverdrup}

\end{frame}

\begin{frame}
  \frametitle{Modification de la phénologie du bloom}
  seulement dans un modèle (même si forcé par observations)

  plot de Mahadevan 2012
\end{frame}

\section{Région subtropicale -- Augmentation de la Chlorophylle dans les fronts}
\sectionframe{1}

\section{Région du Gulf-Stream -- Intensité des fronts}
\sectionframe{2}

\section{Région subpolaire -- Phénologie du bloom}
\sectionframe{3}

\begin{frame}
  \frametitle{Chronométrer le bloom: démarrage et durée}
  \multigraph[width=\textwidth, height=0.8\textheight, keepaspectratio]{9}{phenologie_méthode}
\end{frame}

\begin{frame}
  \frametitle{Différences de phénologie entre background et fronts}
  \multigraph[width=\textwidth, height=0.8\textheight, keepaspectratio]{4}{phenologie_lag}
\end{frame}

\begin{frame}
  \frametitle{Décalage du \emph{\textit{démarrage}} du bloom}
  \multigraph[width=0.95\textwidth]{7}{bloom}

  \begin{overlayarea}{\textwidth}{2\baselineskip}
    \only<6->{fit pour les fronts faibles: \(-6.7 \pm 1.1\) jours}

    \only<7->{fit pour les fronts forts: \(-13.5 \pm 1.5\) jours}
  \end{overlayarea}
\end{frame}

\begin{frame}
  \frametitle{Décalage de la \emph{\textit{durée}} du bloom}
  \includegraphics[
  width=0.90\textwidth, height=0.8\textheight,
  keepaspectratio]{durée_bloom.pdf}

  Blooms plus \emph{longs} dans les \emph{fronts}, mais moins significatif
\end{frame}

\begin{frame}
  C'est la fin
\end{frame}

\end{document}
