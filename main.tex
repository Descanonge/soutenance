
\usepackage{soutenance}

\AtBeginDocument{%
  \bookmark[named=FirstPage, level=subsection]{Title frame}%
}

\title{%
  Impact des Fronts sur le Phytoplancton\\
  dans la Région du Gulf Stream\\
  Quantifié par Imagerie Satellitaire
}

\author{Clément Haëck}

\direction{Marina Lévy et Laurent Bopp}

\institute{%
  Laboratoire d'Océanographie et du Climat\\Expérimentations et Analyses Numériques
}

\begin{document}

%% Title frame
{
  \usebackgroundtemplate{
    \parbox[t][\paperheight]{\paperwidth}{
    \vfill\par
    \hfill
    \includegraphics[width=0.7\paperwidth, height=0.7\paperheight, keepaspectratio]{title_background.jpg}}
  }
  \begin{frame}[plain, noframenumbering]
    \titlepage%
  \end{frame}
}

\section{Introduction}

\begin{frame}
  \frametitle{Qu'est-ce que le phytoplancton ?}
  {
    \centering
    \multigraph{4}{intro_cycles}
  }
\end{frame}

\begin{frame}
  \frametitle{L'importance de la verticale}
  {
    \centering
    \multigraph{4}{intro_verticale}
  }

\end{frame}

\begin{frame}
  \frametitle{Répartition horizontale du phytoplancton}
  \begin{beamercolorbox}[sep=0pt, right]{}
    \includegraphics[width=0.7\textwidth]{composite.jpg}
    \\
    {\footnotesize Image fausses couleurs MODIS-Terra 23-02-2020}
  \end{beamercolorbox}

  \vfill

  \begin{beamercolorbox}[sep=0pt]{}
    Variations à:
    \begin{itemize}
      \item \emph{grandes échelles}: Nord/Sud du Gulf Stream
      \item<+-> \emph{fines échelles}: Tourbillons, filaments, \alert{fronts}
    \end{itemize}
  \end{beamercolorbox}
\end{frame}

\begin{frame}
  \frametitle{Observer le phytoplancton}
  {
    \centering
    \multigraph[width=\textwidth]{4}{levy_2023_fig1}
  }

  \begin{itemize}
    \item<2-> Mesures in-situ
    \item<3-> Modèles numériques biogéochimiques
    \item<4-> \emph{Images satellites}
  \end{itemize}
\end{frame}

\begin{frame}
  \frametitle{Le phytoplancton depuis l'espace}
  schéma explicatif du principe des images sat de chloro
\end{frame}

\begin{frame}
  \frametitle{Influence des courants}
  courants grande échelle: répartition des nutriments + biomes (figure répartition nitrate)

  courants petite échelles
\end{frame}

\subsubsection{}

\begin{frame}
  \frametitle{Déclenchement du bloom}
  \framesubtitle{hypothèse de Sverdrup}
  importance de la profondeur de la couche de mélange

  restratification au printemps

  accompagner d'un schéma ?
\end{frame}

\begin{frame}
  \frametitle{Modification de la phénologie du bloom}
  seulement dans un modèle (même si forcé par observations)

  \includegraphics[width=0.4\textwidth]{mahadevan_2012_fig3.pdf}
  Mahadevan et al.\ (2012)
\end{frame}



\begin{frame}
  \boxtitle{Problématiques}
  \vspace{2em}

  \begin{enumerate}
    \setlength{\itemsep}{1em}
    \item Quantifier la \emph{réponse de la chlorophylle-a} aux dynamiques \emph{frontales} dans la région du Gulf Stream
    \uncover<2->{\item Influence de la \emph{saison}, du \emph{régime} biogéochimique, et de \emph{l’intensité} des fronts ?}
    \uncover<3->{\item Détecter un \emph{bloom précoce} dans les fronts ?}
  \end{enumerate}
\end{frame}

\subsection{Impact des fronts sur le phytoplanctons}
\subsubsection{Upwelling de nutriments}

\begin{frame}
  \frametitle{Frontogénèse et remontée des nutriments}
  \centering
  \multigraph[width=\textwidth, height=0.8\textheight, keepaspectratio]{5}{front_circulation}
\end{frame}

\begin{frame}
  \frametitle{Quantification de l'apport de nutriments}

\end{frame}

\section{Région subtropicale -- Augmentation de la Chlorophylle dans les fronts}
\sectionframe{1}

\begin{frame}
  \frametitle{Détecter les fronts par satellite}
  on veut fronts de densité -> lié à la température -> on utilise la SST

  pleins de méthodes (lister quelques unes)
\end{frame}

\begin{frame}
  \frametitle{Méthode du “Heterogeneity-index”}

  \begin{columns}
    \begin{column}{0.45\textwidth}
      \begin{itemize}
        \item adapté de Liu et Levine (2016)
        \item repose sur une fenêtre glissante
        \item utilise \emph{trois composantes}
              \begin{itemize}
                \item l'écart-type \(\sigma\)
                \item l'asymétrie \(\gamma\)
                \item la bimodalité B
              \end{itemize}
      \end{itemize}
    \end{column}
    \hfill
    \begin{column}{0.45\textwidth}
      \(HI = a \left( b\sigma + c\gamma + dB \right)\)
      \\
      {\small \(a\), \(b\), \(c\), \(d\) étant des coefficients de normalisation}
    \end{column}
  \end{columns}

\end{frame}

\begin{frame}
  \frametitle{Données utilisées}

  ajouter quelques plots pour montrer les diffs

  \begin{itemize}
    \item SST: ESA-SST-CCI / C3S, journalières, 4km, L4
    \item Chl-a: Copernicus-Globcolour, journalières, 4km, L3
  \end{itemize}

\end{frame}

\section{Région du Gulf-Stream -- Intensité des fronts}
\sectionframe{2}

\section{Région subpolaire -- Phénologie du bloom}
\sectionframe{3}


\begin{frame}
  \frametitle{Chronométrer le bloom: démarrage et durée}
  \multigraph[width=\textwidth, height=0.8\textheight, keepaspectratio]{9}{phenologie_méthode}
\end{frame}

\begin{frame}
  \frametitle{Différences de phénologie entre background et fronts}
  \multigraph[width=\textwidth, height=0.8\textheight, keepaspectratio]{4}{phenologie_lag}
\end{frame}

\begin{frame}
  \frametitle{Décalage du \emph{\textit{démarrage}} du bloom}
  \multigraph[width=0.85\textwidth]{7}{bloom}

  \vfill

  \begin{overlayarea}{\textwidth}{2\baselineskip}
    \only<6->{fit pour les fronts faibles: \(-6.7 \pm 1.1\) jours}

    \only<7->{fit pour les fronts forts: \(-13.5 \pm 1.5\) jours}
  \end{overlayarea}
\end{frame}

\begin{frame}
  \frametitle{Décalage de la \emph{\textit{durée}} du bloom}
  \includegraphics[
  width=0.90\textwidth, height=0.7\textheight,
  keepaspectratio]{durée_bloom.pdf}

  \vfill

  Blooms plus \emph{longs} dans les \emph{fronts}, mais moins significatif
\end{frame}

\section{Conclusions}

\begin{frame}
  \frametitle{Conclusions}

\end{frame}

\section{Perspectives}
\begin{frame}
  \frametitle{Perspectives}

  \begin{itemize}
    \item vers le global
    \item vers les PFT (plus que déjà?)
    \item nécessité de développer outils de détection de fronts
  \end{itemize}

\end{frame}


\begin{frame}
  C'est la fin
\end{frame}

\end{document}
